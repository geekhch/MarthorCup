% !Mode:: "TeX:UTF-8"
\documentclass{xcumcmart}
% \title{text}这里是显示在第三页的文章标题
\title{基于线性模型的钢水“脱氧合金化”方案优化}
% \displaytitle{text}这里是显示在承诺书上的文章标题,注意,不能换行,如果题目特别长,要进行适当的缩写
\displaytitle{高教社杯全国大学生数学建模竞赛论文 \LaTeX{} 模板示例}
% \school{text}命令用于在承诺书上显示学校名称。按要求,此处应填写全称
\school{China\TeX{}大学}
% 以下命令分别显示队员及指导教师姓名
\authorone{Ch'en Meng}
\authortwo{Liam Huang}
\authorthree{ShadowInShadow}
\advisor{China\TeX{}}

\usepackage{metalogo,hyperref} % 这里加载的宏包仅仅是为了本示例文档,实际使用时可以根据需要删除。
\begin{document}
\maketitle
\begin{cnabstract}%此处没有采用sbstract命名,是为了将来如果要加入英文摘要时扩展的方便
    本模板根据“2013高教社杯全国大学生数学建模竞赛论文格式规范”中的相关规定制作。供参加
全国大学生数学建模竞赛的同学们使用。Ch'en Meng 制作该模板的目的在于推进竞赛论文格式规范化,
推广 \LaTeX{}。

    本模板将整篇论文视作“文章”(article),而不是“书”(book)。实现时,以 ctexart 为基类。
与 ctexart.cls 的使用基本相同,但\textbf{必须保存为 UTF8 编码并使用\XeLaTeX{}或pdf\LaTeX{}编译}。具体请参考本文源文件。

    为了实现比较好的兼容性,模板仅达到了 \cite{2} 中对竞赛论文的基本要求,而不包括额外的功能,
也没有引入其他的宏包,请使用时根据需要自行添加。

    模板发行包包括如下文件:
    \begin{tabbing}
      {}\qquad\qquad\= cumcmart.cls\qquad \= 模板主文件(必须)\\
      {}\> declaration.tex \> 承诺书及编号页(必须。如有变动,请修改该文件)\\
      {}\> readme.pdf \> 本文件\\
      {}\> readme.tex \> 本文件源文件\\
      {}\> xcumcmart.cls \> 模板文件
    \end{tabbing}

    模板的最新版本可以在 \url{http://liam0205.tk/LaTeX-xcumcmart/} 下载。有关于模板的任何问题,比如 Bug 提交、功能建议、赞助、宣传推广等,均可与 \href{mailto:chenmeng0518+xcumcmart@gmail.com}{Ch'en Meng} 联系。

    感谢 \url{http://www.shumo.com/} 早先以 CCT 方式制作的竞赛论文模板。\\
    \textbf{重要声明:
    \begin{enumerate}
      \item 任何个人或团体可以无限制的自由使用本模板。
      \item 本模板目前尚未得到全国大学生数学建模竞赛组委会的认可,模板作者对使用该模板所引起的后果不负任何责任。
    \end{enumerate}
    }
\end{cnabstract}

%\tableofcontents\newpage%增加目录,要不要都可以。不想要的话,就在本行前加“%”(英文的百分号)
一篇典型的数模竞赛论文,通常包括以下部分,当然,这些并不全是必须的。
\section{问题的重述}
\section{问题分析}
\section{假设与符号}
\section{模型的建立与求解}
    考虑到竞赛论文通常不会太长,所以,预定义好的“定义”、“引理”、“定理”、“命题”、“例”等 5 种
环境的编号都是统一编号,而“推论”的编号,以相应的“定理”编号作为主编号。例子如下:

\begin{definition}
  定义的例子。
\end{definition}

\begin{lemma}
  引理的例子。
\end{lemma}

\begin{theorem}
  定理的例子。
\end{theorem}

\begin{lemma}
  第二个引理的例子。
\end{lemma}

\begin{theorem}
  第二个定理的例子。
\end{theorem}

\begin{corollary}
  推论的例子。
\end{corollary}

\begin{corollary}
  第二个推论的例子。
\end{corollary}

\begin{proposition}
  命题的例子。
\end{proposition}

\begin{example}
  例的例子。
\end{example}

    注意,以上环境的结尾不包括段落结束符,需要根据情况手工添加。比如此处源文件中是空一行作为段落结束。
\section{模型的检验}
\section{进一步讨论}
\section{模型的优缺点}

\begin{thebibliography}{1}
\bibitem{1} 全国大学生数学建模竞赛组委会, 2004 高教社杯全国大学生数学建模竞赛论文格式规范, 2004
\bibitem{2} 全国大学生数学建模竞赛组委会, 2013 高教社杯全国大学生数学建模竞赛论文格式规范, 2013
\end{thebibliography}

\end{document}
